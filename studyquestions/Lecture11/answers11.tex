 \documentclass[11pt]{article}
\author{Andy Black and Andreas Christen}

% settings ========================================================

\usepackage{graphicx}        % graphics
\usepackage{tabularx}        % enhances table layout
\usepackage{titlesec}        % allows customizing header properties

%font settings ====================================================

\titleformat{\section}
{\bfseries\filcenter\Large}{\thesection}{1em}{} 

\titleformat{\subsection}
{\scshape\filcenter}{\thesubsection}{1em}{}

\titleformat{\paragraph}[runin] 
{\normalfont\bfseries} 
{\paragraph.}{.5em}{}[.] 

%lesson numbering =================================================

\begin{document}

\begin{center}
\emph{University of British Columbia, Vancouver}\\
GEOB 300 - Microscale Weather and Climate\\
Knox, \today
\section*{Answers to Study Questions - Lecture 11}
\end{center}

\begin{enumerate}

\item Assume a uniform and linear warming rate of the soil:
\begin{eqnarray*} 
\frac{\Delta T}{\Delta t}&=& \frac{25.3^{\circ}\rm{C} - 24.8^{\circ}\rm{C}}{1\,\rm{h}} \\
&=& 0.5\rm{K}\,\rm{h}^{-1} = 1.38 \times 10^{-4}\rm{K}\,\rm{s}^{-1} 
\end{eqnarray*} 
The heat flux at the surface $Q_{G(\rm{0})}$ is (Lecture 11, Slide 10):
\begin{eqnarray*} 
Q_{G(\rm{0})} &=& Q_{G(\rm{5cm})} + C\, \frac{\Delta T}{\Delta t} \Delta z \\
&=& 25,\rm{W}\,\rm{m}^{-2} + 2 \,\rm{MJ}\,\rm{m}^{-3}\,\rm{K}^{-1}\, \times 1.38 \times 10^{-4}\rm{K}\,\rm{s}^{-1}  \times 0.05\,\rm{m} \\
&=& 25\,\rm{W}\,\rm{m}^{-2} + 13.8\,\rm{W}\,\rm{m}^{-2} \\
&=& \underline{38.8\,\rm{W}\,\rm{m}^{-2}} \\
\end{eqnarray*} 
\item Now we have a cooling rate (i.e. negative change of temperature over time):
\begin{eqnarray*} 
\frac{\Delta T}{\Delta t}&=& \frac{7.0^{\circ}\rm{C} - 7.5^{\circ}\rm{C}}{1\,\rm{h}} \\
&=& -0.5\rm{K}\,\rm{h}^{-1} = -1.38 \times 10^{-4}\rm{K}\,\rm{s}^{-1} 
\end{eqnarray*} 
Similar to the first example, the heat flux at the surface is:
\begin{eqnarray*} 
Q_{G(\rm{0})} &=& Q_{G(\rm{5cm})} + C\, \frac{\Delta T}{\Delta t} \Delta z \\
&=& -12\,\rm{W}\,\rm{m}^{-2} + 2 \,\rm{MJ}\,\rm{m}^{-3}\,\rm{K}^{-1}\, \times \left( -1.38 \times 10^{-4}\rm{K}\,\rm{s}^{-1} \right) \times 0.05\,\rm{m} \\
&=& -12\,\rm{W}\,\rm{m}^{-2} + (-13.88\,\rm{W}\,\rm{m}^{-2})  \\
&=& \underline{-25.9\,\rm{W}\,\rm{m}^{-2}} \\
\end{eqnarray*} 

\item Heat sharing refers to how the soil and the atmosphere (or two other materials) share
in accepting sensible heat (i.e. $Q^{*}$ - $Q_{E}$) during the
daytime and share in releasing it at night. Heat sharing is determined by the ratio of the thermal admittances of the two materials (Lecture 11, slides 27):
\begin{eqnarray*} 
\frac{Q_H}{Q_G} &=& \frac{\mu_a}{\mu_s}
\end{eqnarray*} 
where $\mu_s$ is the thermal admittance of the soil and $\mu_a$ is the thermal admittance of the atmosphere.

\item Rearrange the heat sharing equation from Question 3:
\begin{eqnarray*} 
Q_H &=& Q_G \times \frac{\mu_a}{\mu_s} 
\end{eqnarray*} 
We can calculate $\mu_s$ by:
\begin{eqnarray*} 
\mu_s &=& \sqrt{k\,C} \\
 &=& \sqrt{0.27\,\rm{W}\,\rm{m}^{-1}\,\rm{K}^{-1} \times 2 \times 10^{6} \,\rm{J}\,\rm{m}^{-3}\,\rm{K}^{-1}} \\
 &=& 735 \,\rm{J} \,\rm{m}^{-2} \,\rm{K}^{-1} \,\rm{s}^{-1/2}
\end{eqnarray*} 
Hence:
\begin{eqnarray*} 
Q_H &=& Q_G \times \frac{\mu_a}{\mu_s} \\
&=& 38.8\,\rm{W}\,\rm{m}^{-2} \times \frac{5000 \,\rm{J} \,\rm{m}^{-2} \,\rm{K}^{-1} \,\rm{s}^{-1/2}}{735 \,\rm{J} \,\rm{m}^{-2} \,\rm{K}^{-1} \,\rm{s}^{-1/2}} \\
&=& \underline{264\,\rm{W}\,\rm{m}^{-2}}
\end{eqnarray*} 

\end{enumerate}

\end{document}