\documentclass[11pt]{article}
\author{Andreas Christen}

% settings ========================================================

\usepackage{graphicx}        % graphics
\usepackage{tabularx}        % enhances table layout
\usepackage{titlesec}        % allows customizing header properties
\usepackage{color}

%font settings ====================================================

\titleformat{\section}
{\bfseries\filcenter\Large}{\thesection}{1em}{} 

\titleformat{\subsection}
{\scshape\filcenter}{\thesubsection}{1em}{}

\titleformat{\paragraph}[runin] 
{\normalfont\bfseries} 
{\paragraph.}{.5em}{}[.] 

%lesson numbering =================================================

\begin{document}

\begin{center}
\emph{University of British Columbia, Vancouver}\\
GEOS 300 - Microscale Weather and Climate\\
Knox, updated \today
\section*{Answers to Study Questions - Lecture 2}

\vspace{0.5cm}
\end{center}

\begin{enumerate}

\item The term $\frac{\partial \rho_v}{\partial t} $ describes the change in absolute humidity $\partial \rho_v$ in time $\partial t$ and as such is the \underline{change in storage} within the `volume'. Its unit is partial density of water vapour ($\rm{g}\,\rm{m}^{-3}$) divided by time (in $\rm{s}$), i.e. \underline{$\rm{g}\,\rm{m}^{-3}\,\rm{s}^{-1}$}

\item The term $u \frac{\partial \rho_v}{\partial x}$ describes \underline{transport} of a humidity gradient along the $x$-axis by the wind. Its unit is wind speed ($\rm{m}\,\rm{s}^{-1}$) times partial density of water vapour ($\rm{g}\,\rm{m}^{-3}$) divided by distance (in $\rm{m}$), i.e. again \underline{$\rm{g}\,\rm{m}^{-3}\,\rm{s}^{-1}$}

\item Horizontally homogeneous conditions mean $\frac{\partial \rho_v}{\partial x} = 0$ and $\frac{\partial \rho_v}{\partial y} = 0$. So the conservation equation simplifies to:
\begin{equation}
0 = \frac{\partial \rho_v}{\partial t} + w \frac{\partial \rho_v}{\partial z} 
\end{equation}
or 
\begin{equation}
\frac{\partial \rho_v}{\partial t} = - w \frac{\partial \rho_v}{\partial z} 
\end{equation}
Inserting $\frac{\partial \rho_v}{\partial z} = -1\,\rm{g}\,\rm{m}^{-3}\,\rm{m}^{-1}$ and $w = 0.1 \,\rm{m}\,\rm{s}^{-1}$ results in:
\begin{equation}
\frac{\partial \rho_v}{\partial t} = 0.1\,\rm{g}\,\rm{m}^{-3}\,\rm{s}^{-1}.
\end{equation}
So the `volume' becomes more humid over time.

\end{enumerate}
\end{document}