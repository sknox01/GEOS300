 \documentclass[11pt]{article}
\author{Andy Black and Andreas Christen}

% settings ========================================================

\usepackage{graphicx}        % graphics
\usepackage{tabularx}        % enhances table layout
\usepackage{titlesec}        % allows customizing header properties

%font settings ====================================================

\titleformat{\section}
{\bfseries\filcenter\Large}{\thesection}{1em}{} 

\titleformat{\subsection}
{\scshape\filcenter}{\thesubsection}{1em}{}

\titleformat{\paragraph}[runin] 
{\normalfont\bfseries} 
{\paragraph.}{.5em}{}[.] 

%lesson numbering =================================================

\begin{document}

\begin{center}
\emph{University of British Columbia, Vancouver}\\
GEOB 300 - Microscale Weather and Climate\\
Knox\\
\today
\section*{Study Questions - Lecture 10}
\end{center}

\begin{enumerate}

\item The specific heat of water is $c_p =$ 4.18 kJ kg$^{-1}$ K$^{-1}$. Calculate the heat capacity $C$ of water.

\item Calculate the heat capacity $C$ of a dry mineral soil with a porosity of 55\%. Use values from the table in Lecture 10, Slide 8.

\item Calculate the heat capacity $C$ for the same soil if it is completely saturated.

\item Calculate the heat capacity $C$ of a partly saturated soil with $P = 50\%$, $\theta_a = 0.30$ and an organic to mineral ratio of 1.5. Again, use the table in Lecture 10, Slide 8.

\item If you increase the soil volumetric water content $\theta_w$ of any soil by 0.1, how does the heat capacity $C$ of the soil change?

\item Assume we have a soil with $C = 2\,\rm{MJ}\,\rm{m}^{-3}\,\rm{K}^{-1}$ and we measure a soil heat flux density $Q_G$ of $+100\,\rm{W}\,\rm{m}^{-2}$ all going to the first 10 cm of the soil, how fast would the layer 0-10 cm heat up?

\item Write Fourier's law and explain it briefly.

\item In a dry and uniform mineral soil with a porosity of 55\%, we measure soil temperatures $T_1$ at 2 cm and $T_2$ at 6 cm. $T_1 = 20 \rm{^{\circ}C}$, $T_2 = 18.5 \rm{^{\circ}C}$. Calculate the soil heat flux density $Q_G$ assuming a thermal conductivity of $k = 0.27\,\rm{W}\,\rm{m}^{-1}\,\rm{K}^{-1}$.

\item At 5 cm depth we measure a soil heat flux density $Q_G = 20 \,\rm{W}\,\rm{m}^{-2}$ and simultaneously a temperature gradient of -0.5 K$\,\rm{cm}^{-1}$. Calculate the thermal conductivity $k$.

\item For a soil with a specific heat $c_p =$ 1.8 kJ kg$^{-1}$ K$^{-1}$, a density $\rho =$ 1.4 Mg m$^{-3}$, and a thermal conductivity $k = 0.4\,\rm{W}\,\rm{m}^{-1}\,\rm{K}^{-1}$, calculate the thermal diffusivity $\kappa$.

\item Calculate the thermal admittance $\mu$ for the same soil.

\item Assume we know a soil's dry \underline{mass} fraction of organic material ($f_o = $25\%), and its bulk density ($\rho_s = 1.4 \,\rm{Mg}\,\rm{m}^{-3}$). What is the mass of organic ($M_o$) and mineral material ($M_m$) contained in one cubic metre of this soil?

\item Using the values of specific heat for organic ($c_o$) and mineral ($c_m$) given in the table of Lecture 10, slide 8, calculate the composite heat capacity ($C_s$) for the dry soil in Question 12.

\end{enumerate}

\end{document}