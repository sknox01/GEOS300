 \documentclass[11pt]{article}
\author{Andy Black and Andreas Christen}

% settings ========================================================

\usepackage{graphicx}        % graphics
\usepackage{tabularx}        % enhances table layout
\usepackage{titlesec}        % allows customizing header properties
\usepackage{color}

%font settings ====================================================

\titleformat{\section}
{\bfseries\filcenter\Large}{\thesection}{1em}{} 

\titleformat{\subsection}
{\scshape\filcenter}{\thesubsection}{1em}{}

\titleformat{\paragraph}[runin] 
{\normalfont\bfseries} 
{\paragraph.}{.5em}{}[.] 

%lesson numbering =================================================

\begin{document}

\begin{center}
\emph{University of British Columbia, Vancouver}\\
GEOG 300 - Microscale Weather and Climate\\
Knox
\section*{Answers to Study Questions - Lecture 18}
\end{center}

\begin{enumerate}

\item (a) $\overline{T}$ is the temporal average of $T$: \begin{eqnarray*}
\overline{T} &=& \frac{1}{6} \sum_{t=0}^{5} T(t) \\
&=& \frac{1}{6} \left(12.6 + 11.2 + 11.9 + 13.1 + 12.0 + 11.8 \right) \\
&=& \underline{12.1^{\circ}\textrm{C}}
\end{eqnarray*}
(b) $T^{\prime}$ is the deviation of a single measured value from the temporal average of the time series ($\overline{T}$). For the time step at 40 min we can write:  \begin{eqnarray*}
T(40\textrm{min}) &=& \overline{T^{\prime}} + \overline{T}(40\textrm{min}) \\
T^{\prime}(40\textrm{min}) &=& T(40\textrm{min}) - \overline{T} \\
 &=& 13.1^{\circ}\textrm{C} - 12.1^{\circ}\textrm{C} \\
 &=& \underline{1\, \textrm{K}}\\
\end{eqnarray*}
(c) $T^{\prime 2}$ is the squared deviation for a a single measured value from the temporal average of the time series. For the time step at 20 min we can write: ($\overline{T}$): \begin{eqnarray*}
T^{\prime 2}(20\textrm{min}) &=& (T(20\textrm{min}) - \overline{T})^2 \\
 &=& (11.2^{\circ}\textrm{C} - 12.1^{\circ}\textrm{C})^2 \\
 &=& (-0.9^{\circ}\textrm{C})^2 \\
 &=& \underline{0.81\, \textrm{K}^2}\\
\end{eqnarray*}
(d) $\overline{T^{\prime}}$ is the average deviation in the time series. By definition, this term must be always zero, i.e $\overline{T^{\prime}}=0$ (or more general $\overline{a^{\prime}} = 0$, where $a$ is any parameter or term). Let's check this: \begin{eqnarray*}
\overline{T^{\prime}} &=& \frac{1}{6} \sum_{t=0}^{5} (T(t)-\overline{T}) \\
&=& \frac{1}{6} \left( (12.6-12.1) + (11.2-12.1) + (11.9-12.1) \right. \\
&& \left. + (13.1-12.1) + (12.0-12.1) + (11.8-12.1) \right) \\
&=& \frac{1}{6} \left( 0.5 - 0.9 -0.2 + 1.0 - 0.1 - 0.3 \right) \\
&=& \underline{0\,\textrm{K}}
\end{eqnarray*}
(e) $\overline{T^{\prime 2}}$ is the the variance of the time series of $T$. It is defined as the average of the squared deviations of all time steps. This term is \underline{not} necessarily zero\footnote{Note for those who have taken statistics: You might be used to the \emph{unbiased variance}, where you divide by $N-1$, not $N$:
\begin{eqnarray*}
\overline{T^{\prime 2}} &=& \frac{1}{N-1} \sum_{t=0}^{N-1} (T(t)-\overline{T})^2 = \frac{1}{5} \sum_{t=0}^{5} (T(t)-\overline{T})^2 \\
\end{eqnarray*}
However, in atmospheric turbulence studies, we prefer the biased variance (divided by $N$, as shown in the question above and the lecture slides where we divide by $N$). The biased variance is a good estimation of the dispersion of a sample of observations, but not necessarily the best measure of the whole population of possible observations.
}

\begin{eqnarray*}
\overline{T^{\prime 2}} &=& \frac{1}{6} \sum_{t=0}^{5} (T(t)-\overline{T})^2 \\
&=& \frac{1}{6} \left( (12.6-12.1)^2 + (11.2-12.1)^2 + (11.9-12.1) ^2\right. \\
&& \left. + (13.1-12.1)^2 + (12.0-12.1)^2 + (11.8-12.1)^2 \right) \\
&=& \frac{1}{6} \left( 0.25 + 0.81 + 0.04 + 1 + 0.01 + 0.09 \right) \\
&=& \underline{0.37\,\textrm{K}^2}
\end{eqnarray*}
(f) $\overline{T^{\prime}}^2$ is the temporal average of the deviations squared. Note the fine (but essential) difference of the overbar that does not include the square, and because $\overline{T^{\prime}} = 0$ (see (d)) also its square is zero:
\begin{eqnarray*}
\overline{T^{\prime}}^2 &=& \overline{T^{\prime}} \times \overline{T^{\prime}} \\
&=& 0\,\textrm{K} \times 0\,\textrm{K} \\
&=& \underline{0\,\textrm{K}^2}
\end{eqnarray*}

\item (a) The average of a constant is equal the constant itself
\begin{eqnarray*}
\overline{5} = \frac{1}{1}\sum_{i=0}^{0} 5 = 5
\end{eqnarray*}
(b) The average of a constant times a value is equal the constant times the averaged value:
\begin{eqnarray*}
\overline{8v} = 8 \times \overline{v}
\end{eqnarray*}
(c) Similarly we can regard temporal averages as constants (they are not changing over time): 
\begin{eqnarray*}
\overline{\overline{T}\overline{p}} = \overline{T}\overline{p}
\end{eqnarray*}
(d) The average of an averaged value is equal the averaged value
\begin{eqnarray*}
\overline{\overline{u}} = \overline{u}
\end{eqnarray*}
(e) Similar to 1(d):
\begin{eqnarray*}
\overline{q^{\prime}} = 0 
\end{eqnarray*}
(f) Again, the first term is zero - so is it's product:
\begin{eqnarray*}
\overline{T^{\prime}} \times \overline{w} = 0 \times \overline{w} = 0
\end{eqnarray*}
(g)
\begin{eqnarray*}
\overline{3q^{\prime}} = 3 \times \overline{q^{\prime}} = 3 \times 0 = 0
\end{eqnarray*}
(h)
\begin{eqnarray*}
\overline{w^{\prime} \times \overline{u}} = \overline{w^{\prime}} \times \overline{u} = 0 \times \overline{u} = 0 
\end{eqnarray*}
(i) Combining the results from (c) and (e): 
\begin{eqnarray*}
\overline{\overline{T}p} &=& \overline{T} \times \overline{(\overline{p}+p^{\prime})} \\
&=& \overline{T}\overline{p} + \overline{T} \times \underbrace{\overline{p^{\prime}}}_{=0} \\
&=& \overline{T}\overline{p}
\end{eqnarray*}
(j) In a first step apply Reynold's decomposition to both, $w$ and $T$. One of the resulting terms is a covariance ($\overline{w^{\prime}T^{\prime}}$) which is not zero (see lecture):
\begin{eqnarray*}
\overline{wT} &=& \overline{(\overline{w}+w^{\prime}) \times (\overline{T}+T^{\prime})} \\
&=& \overline{\overline{w}\overline{T}} + \underbrace{\overline{\overline{w} T^{\prime}}}_{=0} + \underbrace{\overline{w^{\prime}\,\overline{T}}}_{=0}  +\overline{w^{\prime}T^{\prime}}\\
&=& \overline{w}\overline{T} + \underbrace{\overline{w^{\prime}T^{\prime}}}_{\textrm{Covariance}}
\end{eqnarray*}

\item Calculate the following parameters if $\overline{u} = 4\, \textrm{m}\, \textrm{s}^{-1}$, $\overline{v} = 0\, \textrm{m}\, \textrm{s}^{-1}$, $\overline{w} = 0\, \textrm{m}\, \textrm{s}^{-1}$, $\sigma_u = 0.4\, \textrm{m}\, \textrm{s}^{-1}$, $\sigma_v = 0.2\, \textrm{m}\, \textrm{s}^{-1}$, and $\sigma_w = 0.1\, \textrm{m}\, \textrm{s}^{-1}$.

(a) $I_u$ is the turbulence intensity of the longitudinal wind component $u$. $I_u$ has no unit.
\begin{eqnarray*}
I_u &=& \sigma_u / \overline{u} \\
&=& 0.4\,\textrm{m}\,\, \textrm{s}^{-1} / 4 \,\textrm{m}\,\, \textrm{s}^{-1} \\
&=& \underline{0.1}
\end{eqnarray*}
(b) $I_w$ is the turbulence intensity of the vertical wind component $w$. $I_w$ has no unit.
\begin{eqnarray*}
I_w &=& \sigma_w / \overline{u} \\
&=& 0.1\,\textrm{m}\,\, \textrm{s}^{-1} / 4 \,\textrm{m}\,\, \textrm{s}^{-1} \\
&=& \underline{0.025}
\end{eqnarray*}
(c) $\overline{w^{\prime 2}}$ is the variance of the vertical wind component $w$. The variance is the square of the standard deviation $\sigma_w$:
\begin{eqnarray*}
\overline{w^{\prime 2}} &=& \sigma_w^2  \\
&=& (0.1 \,\textrm{m}\,\, \textrm{s}^{-1})^2 \\
&=& \underline{0.01 \,\textrm{m}^2\,\, \textrm{s}^{-2}}
\end{eqnarray*}
(d) 
\begin{eqnarray*}
\overline{u^{\prime 2}} + \overline{v^{\prime 2}} + \overline{w^{\prime 2}} &=& \sigma_u^2 + \sigma_v^2 + \sigma_w^2  \\
&=& (0.4 \,\textrm{m}\,\, \textrm{s}^{-1})^2 + (0.2 \,\textrm{m}\,\, \textrm{s}^{-1})^2 + (0.1 \,\textrm{m}\,\, \textrm{s}^{-1})^2 \\
&=& 0.16 \,\textrm{m}^2\,\, \textrm{s}^{-2} + 0.04 \,\textrm{m}^2\,\, \textrm{s}^{-2} + 0.01 \,\textrm{m}^2\,\, \textrm{s}^{-2} \\
&=& \underline{0.21\, \textrm{m}^2\,\, \textrm{s}^{-2}}
\end{eqnarray*}
(e) $MKE / m$ is the mean kinetic energy per unit mass (lecture 18, slide 15), i.e.
\begin{eqnarray*}
MKE / m &=& \frac{1}{2}(\overline{u}^2+\overline{v}^2+\overline{w}^2)\\
&=& \frac{1}{2}(4^2+0^2+0^2)\\
&=& \frac{1}{2}(16)\\
&=& \underline{8\, \textrm{m}^2\,\, \textrm{s}^{-2}}\\
\end{eqnarray*}
(f) $\overline{e}$ is the turbulent kinetic energy per unit mass (lecture 18, slide 15) i.e.
\begin{eqnarray*}
\overline{e} &=&  \frac{1}{2}(\overline{u^{\prime 2}}+\overline{v^{\prime 2}}+\overline{w^{\prime 2}})\\
&=& \frac{1}{2}(\sigma_u^2+\sigma_v^2+\sigma_w^2)\\
&=& \frac{1}{2}\left( (0.4 \,\textrm{m}\,\, \textrm{s}^{-1})^2 + (0.2 \,\textrm{m}\,\, \textrm{s}^{-1})^2 + (0.1 \,\textrm{m}\,\, \textrm{s}^{-1})^2 \right) \\
&=& \frac{1}{2}(0.16 \,\textrm{m}^2\,\, \textrm{s}^{-2} + 0.04 \,\textrm{m}^2\,\, \textrm{s}^{-2} + 0.01 \,\textrm{m}^2\,\, \textrm{s}^{-2}) \\
&=& \underline{0.105\, \textrm{m}^2\,\, \textrm{s}^{-2}}
\end{eqnarray*}

\end{enumerate}


\noindent

\end{document}