 \documentclass[11pt]{article}
\author{Andy Black and Andreas Christen}

% settings ========================================================

\usepackage{graphicx}        % graphics
\usepackage{tabularx}        % enhances table layout
\usepackage{titlesec}        % allows customizing header properties

%font settings ====================================================

\titleformat{\section}
{\bfseries\filcenter\Large}{\thesection}{1em}{} 

\titleformat{\subsection}
{\scshape\filcenter}{\thesubsection}{1em}{}

\titleformat{\paragraph}[runin] 
{\normalfont\bfseries} 
{\paragraph.}{.5em}{}[.] 

%lesson numbering =================================================

\begin{document}

\begin{center}
\emph{University of British Columbia, Vancouver}\\
GEOB 300 - Microscale Weather and Climate\\
Knox\\
\today
\section*{Study Questions - Lecture 12}
\end{center}

\begin{enumerate}

\item How does the amplitude and phase lag of soil temperatures in a completely dry soil depend on (a) depth, and (b) soil organic matter
content (at constant porosity)?

\item Calculate the angular frequency for a diurnal $\omega_d$ and an annual temperature wave $\omega_a$.

\item Calculate the daily and annual damping depth $D$ for a mineral soil with a thermal diffusivity $\kappa = 5.0 \times 10^{-7}\,\rm{m}^2\,\rm{s}^{-1}$. How do you interpret your results?

\item For the same soil, calculate when the soil temperature reaches its maximum at 5 cm if the maximum surface temperature is measured at 13:00?

\item In a different soil, you measure the daily maximum 5-cm soil temperature at 15:00 and the maximum 15-cm soil temperature at 18:00. Assume homogeneous soil properties and sinusoidal waves, and calculate the soil's thermal diffusivity $\kappa$.

\item Briefly compare the temperature regimes of mineral and organic soils.

\end{enumerate}
\end{document}