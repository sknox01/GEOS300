 \documentclass[11pt]{article}
\author{Andy Black and Andreas Christen}

% settings ========================================================

\usepackage{graphicx}        % graphics
\usepackage{tabularx}        % enhances table layout
\usepackage{titlesec}        % allows customizing header properties

%font settings ====================================================

\titleformat{\section}
{\bfseries\filcenter\Large}{\thesection}{1em}{} 

\titleformat{\subsection}
{\scshape\filcenter}{\thesubsection}{1em}{}

\titleformat{\paragraph}[runin] 
{\normalfont\bfseries} 
{\paragraph.}{.5em}{}[.] 

%lesson numbering =================================================

\begin{document}

\begin{center}
\emph{University of British Columbia, Vancouver}\\
GEOS 300 - Microscale Weather and Climate\\
Knox
\section*{Study Questions - Lecture 22}
\end{center}

\begin{enumerate}

\item You measure a covariance $\overline{w^{\prime}T^{\prime}} = -0.031\,\rm{m}\,\rm{s}^{-1}\,\rm{K}$. Average air temperature is 10$^{\circ}$C. Calculate $Q_H$. Is this a day-time or night-time situation?

\item You measure a covariance $\overline{w^{\prime}\rho_v^{\prime}} = 1.73 \times 10^{-4}\, \rm{kg}\,\rm{m}^{-2}\,\rm{s}^{-1}$. $\rho_v$ is the water vapour density in $\rm{kg}\,\rm{m}^{-3}$. Average air temperature is 30$^{\circ}$C. Calculate $Q_E$.

\item Determine the Bowen ratio $\beta$ if $\overline{w^{\prime}T^{\prime}} = 0.121\,\rm{m}\,\rm{s}^{-1}\,\rm{K}$ and $\overline{w^{\prime}\rho_v^{\prime}} = 1.21 \times 10^{-4}\, \rm{kg}\,\rm{m}^{-2}\,\rm{s}^{-1}$. Average air temperature is 20$^{\circ}$C.

\item Given is $Q_E$ = 240 W m$^{-2}$ at 20$^{\circ}$C air temperature. Determine the covariance $\overline{w^{\prime}q^{\prime}}$, where $q$ is the specific humidity (in g water vapour per kg air ,i.e. g kg$^{-1}$).

\item Over a rice paddy you measure a covariance between vertical wind and methane concentration $\rho_{\rm{CH}_4}$ in $\rm{\mu g}\,\rm{m}^{-3}$ of  $\overline{w^{\prime}\rho_{\rm{CH}_4}^{\prime}} = 10 \,\rm{m}\,\rm{s}^{-1}\,\rm{\mu g}\,\rm{m}^{-3}\,$. Determine the mass flux density between surface and atmosphere.

\end{enumerate}

\noindent

\end{document}