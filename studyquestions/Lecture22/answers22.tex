 \documentclass[11pt]{article}
\author{Andy Black and Andreas Christen}

% settings ========================================================

\usepackage{graphicx}        % graphics
\usepackage{tabularx}        % enhances table layout
\usepackage{titlesec}        % allows customizing header properties
\usepackage{color}

%font settings ====================================================

\titleformat{\section}
{\bfseries\filcenter\Large}{\thesection}{1em}{} 

\titleformat{\subsection}
{\scshape\filcenter}{\thesubsection}{1em}{}

\titleformat{\paragraph}[runin] 
{\normalfont\bfseries} 
{\paragraph.}{.5em}{}[.] 

%lesson numbering =================================================

\begin{document}

\begin{center}
\emph{University of British Columbia, Vancouver}\\
GEOS 300 - Microscale Weather and Climate\\
Knox
\section*{Answers to Study Questions - Lecture 22}
\textcolor{black}{\today}
\end{center}

\begin{enumerate}

\item Read the air density $\rho_a$ from Oke `Boundary Layer Climates' Table A3.1.
\begin{eqnarray*}
Q_H &=& C_a \, \overline{w^{\prime}T^{\prime}} = \rho_a \, c_p \, \overline{w^{\prime}T^{\prime}} \\
&=& 1.230\,\rm{kg}\,\rm{m}^{-3} \times 1010 \,\rm{J}\,\rm{kg}^{-1}\,\rm{K}^{-1} \times -0.031 \,\rm{m}\,\rm{s}^{-1}\,\rm{K} \\
&=& -38.5 \,\rm{J}\,\rm{m}^{-2}\,\rm{s}^{-1} =  \underline{ -38.5\,\rm{W}\,\rm{m}^{-2}}
\end{eqnarray*}
This is likely a night-time situation, when the surface is cold and that atmosphere is warmer. The profile of potential temperature is likely increasing with height. Excess sensible heat (i.e.\ $\rho_a c_p T$) is transported from higher levels of the atmosphere down to the layers close to the surface, as indicated by the negative sign of $Q_H$. 

\item Use the latent heat of vaporization for water, $L_v$, form Oke `Boundary Layer Climates' Table A3.1.
\begin{eqnarray*}
Q_E &=& L_v \, \overline{w^{\prime}\rho_v^{\prime}} \\
&=& 2.432 \times 10^{6} \,\rm{J}\,\rm{kg}^{-1} \times 1.73 \times 10^{-4}\, \rm{kg}\,\rm{m}^{-2}\,\rm{s}^{-1} \\
&=& 420 \,\rm{J}\,\rm{m}^{-2}\,\rm{s}^{-1} =  \underline{420\,\rm{W}\,\rm{m}^{-2}}
\end{eqnarray*}

\item The bowen ratio is $\beta = Q_H / Q_E$. Analogous to Question 1, calculate $Q_H$:
\begin{eqnarray*}
Q_H &=& \rho_a \, c_p \, \overline{w^{\prime}T^{\prime}} \\
&=& 1.188\,\rm{kg}\,\rm{m}^{-3} \times 1010 \,\rm{J}\,\rm{kg}^{-1}\,\rm{K}^{-1} \times 0.121 \,\rm{m}\,\rm{s}^{-1}\,\rm{K} \\
&=& 145.1  \,\rm{W}\,\rm{m}^{-2}
\end{eqnarray*}
Analogous to Question 2, calculate $Q_E$:
\begin{eqnarray*}
Q_E &=& L_v \, \overline{w^{\prime}\rho_v^{\prime}} \\
&=& 2.453 \times 10^{6} \,\rm{J}\,\rm{kg}^{-1} \times 1.21 \times 10^{-4}\, \rm{kg}\,\rm{m}^{-2}\,\rm{s}^{-1} \\
&=& 296.8 \,\rm{W}\,\rm{m}^{-2}
\end{eqnarray*}
Hence 
\begin{eqnarray*}
\beta &=& \frac{Q_H}{Q_E} = \frac{145.1\,\rm{W}\,\rm{m}^{-2}}{296.8\,\rm{W}\,\rm{m}^{-2}} = \underline{0.49}
\end{eqnarray*}

\item The relation between specific humidity $q$ and water vapour density $\rho_v$ is: $\rho_v = \rho_a q$, where $\rho_a$ is the density of the (moist) air. Because $\rho_a$ is not changing significantly it can be taken out of the averaging operator:
\begin{eqnarray*}
Q_E &=& L_v \, \overline{w^{\prime}\rho_v^{\prime}} \\
    &=& \, L_v\, \overline{w^{\prime}(\rho_a q)^{\prime}} \\
    &=& \rho_a \, L_v\, \overline{w^{\prime}q^{\prime}}
\end{eqnarray*}
rearrange:
\begin{eqnarray*}
\overline{w^{\prime}q^{\prime}} &=& \frac{Q_E}{\rho_a \, L_v}\\
&=& \frac{240\,\rm{J}\,\rm{s}^{-1}\,\rm{m}^{-2}}{1.188\,\rm{kg}\,\rm{m}^{-3}  \times 2.453 \times 10^{6} \,\rm{J}\,\rm{kg}^{-1} } \\
&=& 8.24 \times 10^{-5}\,\rm{m}\,\rm{s}^{-1}\,\rm{kg}\,\rm{kg}^{-1} \\
&=& \underline{0.0824\,\rm{m}\,\rm{s}^{-1}\,\rm{g}\,\rm{kg}^{-1}}
\end{eqnarray*}

\item This covariance is already a mass flux density, because its units are mass per square meter and per second if sorted properly:
\begin{eqnarray*}
F_{\rm{CH}_4} &=& \overline{w^{\prime}\rho_{\rm{CH}_4}^{\prime}} \\
 &=& 10 \,\rm{m}\,\rm{s}^{-1}\,\rm{\mu g}\,\rm{m}^{-3}\\
 &=& \underline{10 \,\rm{\mu g}\,\rm{m}^{-2}\,\rm{s}^{-1}}\\
\end{eqnarray*}

\end{enumerate}
\noindent

\end{document}